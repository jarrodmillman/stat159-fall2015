% Document settings
\documentclass[11pt]{article}
\usepackage[margin=1in]{geometry}
\usepackage[pdftex]{graphicx}
\usepackage{multirow}
\usepackage{setspace}
\usepackage{hyperref}
\pagestyle{plain}
\setlength\parindent{0pt}

\begin{document}

% Course information
\begin{tabular}{ l l }
  \multirow{3}{*}{\includegraphics[height=1.25in,width=1.25in]{../_fig/ucberkeleyseal_874_540.eps}}
  & \LARGE Statistics 159 \& 259 --- Fall 2015 Syllabus\\
  & \LARGE Reproducible and Collaborative Statistical Data Science \\\\
  & \Large CCN: 87680 (Stat 159) and 87812 (Stat 259)\\
  & \Large Class meets TuTh 9:30--11A in 150 GSPP \\
  & \Large Lab meets M 10--12P or 12--2P in 340 EVANS \\\\
\end{tabular}
\vspace{10mm}

% Professor information
\begin{tabular}{ l l }
  \multirow{6}{*} & \large K. Jarrod Millman \\
  & \large http://www.jarrodmillman.com \\
  & \large Office Location: 210 Barker Hall \\
  & \large Office Hours: TBD \\
\end{tabular}
\vspace{5mm}
\begin{center} I reserve the right to make changes to the syllabus.\\
\end{center}

% Course details
\textbf {\large \\ Course Description:}
A project-based introduction to statistical data science. Through lectures,
computational laboratories, readings, homeworks, and a group project, you will learn
practical techniques and tools for producing statistically sound and
appropriate, reproducible, and verifiable computational answers to scientific
questions.  Course emphasizes version control, testing, process automation,
code review, and collaborative programming. Software tools include Bash, Git,
Python, and \LaTeX.

\textbf {Prerequisites:} Statistics 133, Statistics 134, and Statistics 135
(or equivalent). Graduate standing is required to register for Statistics 259.

\textbf {Credit Hours:} 4 \\

\textbf {\large Text(s):} 
Readings will be assigned weekly and will mostly consist of articles and tutorials. \\

\textbf {\large Course Objectives:} \\
At the completion of this course, students will:
\begin{enumerate} \itemsep-0.4em
  \item be proficient at the Unix commandline
  \item be expert at version control with Git
  \item be able to write documents in Markdown or \LaTeX (including using
        pandoc)
  \item be familiar with scientific computing in Python
  \item understand the computational and statistical issues involved with
        reproducibility
  \item be familiar with computational issues in modern statistical data
        analysis through hands-on analysis of functional MRI data
\end{enumerate}

\textbf {\large Grading:} \\
\hspace*{40mm}
\begin{tabular}{ l l }
Reading  & 10\% \\
Quiz  & 15\% \\
Homework & 20\% \\
Project  & 55\%
\end{tabular} \\\\

For each assigned reading, you will submit a 2 paragraph report
by 21:00 on the Thursday it is due.
The first paragraph should summarize the reading.  The second paragraph
should briefly explore something that interested you (e.g., you
may wish to focus on one aspect of the paper in more depth, you
may wish to discuss something in the reading that you disagree
with).

Quizzes will be held during class or lab.  I will drop your
lowest score.

There will be 2 homeworks (to be submitted by 21:00 on the Thursday it is
due), which will involve a substantial amount of
effort.  You may discuss the homework with your classmates, but you will
be required to work on the homework independently.

The majority of the class focuses on a final group project, which is
explained in more detail here:
 \url{http://www.jarrodmillman.com/stat159-fall2015/pages/project.html}\\

%\newpage

\textbf{\large Course Policies:}

\textbf{Attendance and behavior in class}: You are expected to attend all lectures
and labs.  Any known or potential extracurricular conflicts should be discussed
in person with me during the first two weeks of the semester, or as
soon as they arise. \textbf{Cellphones} are to be turned off during class time.
\textbf{Laptop} use during class will often be required, but should be
used for course work only (i.e., not for surfing the web).\\

\textbf{Submission of assignments}: Assignments will be accepted by electronic
submission to GitHub only.  There will be no makeup quizzes. No
late reading reports or homeworks will be accepted. \\ % Grades of Incomplete will be granted
%only for dire medical or personal emergencies that cause you to miss the final project
%presentation, and only if your work up to that point has been satisfactory.\\

\textbf{Academic integrity}: Any test, paper, or report submitted by you is presumed
to be your own original work that has not previously been submitted for credit
in another course. While you are encouraged to work together on homework
assignments, the work and writeup must be your own. For example, suggesting a
function to another student is acceptable, whereas simply giving him or her
your own code is not.  If you are not clear about the expectations for
completing an assignment or taking a quiz, be sure to seek clarification from
me or GSI beforehand. Any evidence of cheating and plagiarism will
be subject to disciplinary action.  Please read the Honor
Code (\url{http://asuc.org/honorcode/index.php}) carefully.\\

\textbf{Class discussion}: 
Rather than emailing questions to the teaching staff, you should post
your questions on Piazza. Please read Eric Raymond's ``How To Ask Questions
The Smart Way'' (\url{http://www.catb.org/esr/faqs/smart-questions.html}).

Find our class page at: \url{https://piazza.com/berkeley/fall2015/stat159/home}\\

\textbf{Students with disabilities}: If you need accommodations, please make
arrangements in at timely manner through DSP.\\

\noindent\textbf{Important Dates}:
\begin{center} \begin{minipage}{5in}
\begin{flushleft}
Form teams \dotfill Sept. 17\\
Homework 1 \dotfill Sept. 24\\
Project proposal \dotfill Oct. 1\\
Homework 2 \dotfill Oct. 22\\
Progress presentation \dotfill Nov. 3 \& 5\\
Draft report \dotfill Nov. 12\\
Project presentation \dotfill Dec. 1 \& 3\\
Final report \dotfill Dec. 14\\
\end{flushleft}
\end{minipage}
\end{center}

\newpage

% Course Outline
\textbf {\large Tentative Course Outline}:

The weekly coverage might change as it depends on the progress of the class.
%However, you must keep up with the reading assignments.

\begin{table}[h!]
\normalsize % The size of the table text can be changed depending on content. Remove if desired.
\begin{tabular}{ | c | c | }
\hline
\textbf{Week} & \textbf{Content} \\
\hline
Week 1 & \begin{minipage}{.85\textwidth}
\begin{itemize} \itemsep-0.4em
	\vspace{1mm}
	\item Course introduction
	\vspace{1mm}
\end{itemize}
\end{minipage} \\
\hline
Week 2 & \begin{minipage}{.85\textwidth}
\begin{itemize} \itemsep-0.4em
	\vspace{1mm}
	\item Introduction to Git; Using the bash shell
	\item \textbf{Reading 1}: L Preeyanon, AB Pyrkosz, and CT Brown.
              ``Reproducible bioinformatics research for biologists.''
              %Implementing Reproducible Research (2014)
              (2014)
	\vspace{1mm}
\end{itemize}
\end{minipage} \\
\hline
Week 3 & \begin{minipage}{.85\textwidth}
\begin{itemize} \itemsep-0.4em
	\vspace{1mm}
	\item Statistical analysis of fMRI
	\item \textbf{Reading 2}: MA Lindquist. ``The statistical analysis of fMRI data.''
              %Statistical Science 23.4 (2008)
              (2008)
	\vspace{1mm}
\end{itemize}
\end{minipage} \\
\hline
Week 4 & \begin{minipage}{.85\textwidth}
\begin{itemize} \itemsep-0.4em
	\vspace{1mm}
	\item Introduction to Python
	\item \textbf{Reading 3}: F P\'{e}rez, BE Granger, and JD Hunter.
              ``Python: an ecosystem for scientific computing.''
              %Computing in Science \& Engineering 13.2 (2011)
              (2011)
        \item \textbf{Form teams}
	\vspace{1mm}
\end{itemize}
\end{minipage} \\
\hline
Week 5 & \begin{minipage}{.85\textwidth}
\begin{itemize} \itemsep-0.4em
	\vspace{1mm}
	\item Scientific computing with Python
	\item \textbf{Homework 1} 
	\vspace{1mm}
\end{itemize}
\end{minipage} \\
\hline
Week 6 & \begin{minipage}{.85\textwidth}
\begin{itemize} \itemsep-0.4em
	\vspace{1mm}
	\item Collaborative workflow with Git
	\item \textbf{Project proposal}
	\vspace{1mm}
\end{itemize}
\end{minipage} \\
\hline
Week 7 & \begin{minipage}{.85\textwidth}
\begin{itemize} \itemsep-0.4em
	\vspace{1mm}
	\item Exploratory data analysis
	\item \textbf{Reading 4}: JB Buckheit and DL Donoho.
              ``Wavelab and reproducible research.'' 1995.
	\vspace{1mm}
\end{itemize}
\end{minipage} \\
\hline
Week 8 & \begin{minipage}{.85\textwidth}
\begin{itemize} \itemsep-0.4em
	\vspace{1mm}
	\item Project organization, process automation
	\item \textbf{Reading 5}: KJ Millman and F P\'{e}rez.
              ``Developing open source scientific practice.''
              %Implementing Reproducible Research (2014)
              (2014)
	\vspace{1mm}
\end{itemize}
\end{minipage} \\
\hline
Week 9 & \begin{minipage}{.85\textwidth}
\begin{itemize} \itemsep-0.4em
	\vspace{1mm}
	\item Statistical analysis
	\item \textbf{Homework 2}
	\vspace{1mm}
\end{itemize}
\end{minipage} \\
\hline
Week 10 & \begin{minipage}{.85\textwidth}
\begin{itemize} \itemsep-0.4em
	\vspace{1mm}
	\item TBD % Statistical analysis II
	\vspace{1mm}
\end{itemize}
\end{minipage} \\
\hline
Week 11 & \begin{minipage}{.85\textwidth}
\begin{itemize} \itemsep-0.4em
        \vspace{1mm}
        \item \textbf{Project progress presentation}
        \vspace{1mm}
\end{itemize}
\end{minipage} \\
\hline
Week 12 & \begin{minipage}{.85\textwidth}
\begin{itemize} \itemsep-0.4em
	\vspace{1mm}
	\item TBD % Scientific computing with Python II
        \item \textbf{Draft report}
	\vspace{1mm}
\end{itemize}
\end{minipage} \\
\hline
Week 13 & \begin{minipage}{.85\textwidth}
\begin{itemize} \itemsep-0.4em
	\vspace{1mm}
        \item TBD % Model selection/validation, Selective inference
	\vspace{1mm}
\end{itemize}
\end{minipage} \\
\hline
Week 14 & \begin{minipage}{.85\textwidth}
\begin{itemize} \itemsep-0.4em
	\vspace{1mm}
	\item TBD % Final thoughts
	\vspace{1mm}
\end{itemize}
\end{minipage} \\
\hline
Week 15 & \begin{minipage}{.85\textwidth}
\begin{itemize} \itemsep-0.4em
        \vspace{1mm}
        \item \bf{Project presentation}
        \vspace{1mm}
\end{itemize}
\end{minipage} \\
\hline
Week 16 & \begin{minipage}{.85\textwidth}
\begin{itemize} \itemsep-0.4em
        \vspace{1mm}
        \item \bf{RR Week}
        \vspace{1mm}
\end{itemize}
\end{minipage} \\
\hline
Week 17 & \begin{minipage}{.85\textwidth}
\begin{itemize} \itemsep-0.4em
        \vspace{1mm}
        \item \bf{Final report due Monday}
        \vspace{1mm}
\end{itemize}
\end{minipage} \\
\hline
\end{tabular} 
\end{table}

\end{document}
