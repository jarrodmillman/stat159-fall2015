% Document settings
\documentclass[11pt]{article}
\usepackage[margin=1in]{geometry}
\usepackage[pdftex]{graphicx}
\usepackage{multirow}
\usepackage{setspace}
\usepackage{hyperref}
\pagestyle{plain}
\setlength\parindent{0pt}

\begin{document}

% Course information
\begin{tabular}{ l l }
  \multirow{3}{*}{\includegraphics[height=1.25in,width=1.25in]{../fig/ucberkeleyseal_874_540.eps}}
  & \LARGE Statistics 159 \& 259 --- Fall 2015 Project\\
  & \LARGE Reproducible and Collaborative Statistical Data Science \\\\
  & \begin{minipage}{5in}
\begin{flushleft}
Form teams \dotfill Sept. 17\\
Project proposal \dotfill Oct. 1\\
Progress report \dotfill Nov. 3 \& 5\\
Project presentation \dotfill Dec. 1 \& 3\\
Project report \dotfill Dec. 18\\
\end{flushleft}
\end{minipage}
\end{tabular}

\vspace{10mm}

% Course details

\textbf {\large \\Learning objectives:} Working with complex and large datasets; convolution
(hemodynamic modeling, smoothing); interpolation (slice time correction, image
resampling); optimization (registration, advanced statistics); basic linear
algebra (statistics).


\textbf {\large \\ Project overview:}

This semester the group project will involve reproducing a published result
using the analysis of functional Magnetic Resonance Imaging (MRI). 
Functional MRI (fMRI) allows scientists to localize which parts
of the brain are associated with specific cognitive tasks. There is no
expectation that you will have a background in neuroscience. You will learn
everything you need to know about the method in the course. The intention of
focusing on fMRI data is merely to provide a concrete problem domain that
exemplifies the types of programming and statistical challenges present in many
modern statistical applications. Additionally, the weekly labs will prepare
you---through a series of graded exercises---with the skills and background you
will need for the group project.

\textbf {\large \\ Forming teams:}
Teams will need to be decided by \textbf{Thursday, September 17th}.  Each team
should consist of 4 students.  Students enrolled in 159 (259) will be required to
work with students enrolled in 159 (259).  You should ensure that
your team is composed of individuals with different strengths.  For example,
each team will need people with strong computational as well as statistical skills.

\textbf {\large \\ Project proposal:}
Each team will need to submit a project proposal on \textbf{Thursday, October 1st}.
The proposal should be written in \LaTeX and submitted to your team's GitHub
repository.  Use this template \texttt{proposal.tex}.

\textbf {\large \\ Progress report:}
Each team will need to submit a written progress report before class on
\textbf{Tuesday, November 3rd}.

Additionally, each team will need to present a 5 minute progress report
on \textbf{Tuesday, November 3rd} or \textbf{Thursday,  November 5th}.

\textbf {\large \\ Project presentation:}

\textbf {\large \\ Final report:}
Each team will need to submit a final written report by 5P on
\textbf{Monday, December XXX}.  For more information see the
template.

\textbf {\large \\ Potential papers:}

OpenfMRI \url{https://www.openfmri.org/} is a good resources for publicly-available
fMRI datasets.

Here is a list of potential papers and suggested topics:

\begin{itemize}
\item Haxby et al. (2001): Faces and Objects in Ventral Temporal Cortex (fMRI)
      
      use the data set to show regions more involved in face processing versus
      scenes/ object processing - check that this about the region that was used to
      predict the condition in this paper ... 

      For more information see: \url{http://dev.pymvpa.org/datadb/haxby2001.html}

\item ...
\end{itemize}

\textbf {\large \\ Additional resources:}

\begin{itemize}
\item \url{https://www.coursera.org/course/fmri}
\end{itemize}

\end{document}
