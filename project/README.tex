% Document settings
\documentclass[11pt]{article}
\usepackage[margin=1in]{geometry}
\usepackage[pdftex]{graphicx}
\usepackage{multirow}
\usepackage{setspace}
\usepackage{hyperref}
\pagestyle{plain}
\setlength\parindent{0pt}

\begin{document}

% Course information
\begin{tabular}{ l l }
  \multirow{3}{*}{\includegraphics[height=1.25in,width=1.25in]{../_fig/ucberkeleyseal_874_540.eps}}
  & \LARGE Statistics 159 \& 259 --- Fall 2015 Project\\
  & \LARGE Reproducible and Collaborative Statistical Data Science \\\\
  & \begin{minipage}{5in}
\begin{flushleft}
Form teams \dotfill Sept. 22\\
Project proposal \dotfill Oct. 1\\
Progress presentation \dotfill Nov. 3 \& 5\\
Draft report \dotfill Nov. 12\\
Project presentation \dotfill Dec. 1 \& 3\\
Final report \dotfill Dec. 14\\
\end{flushleft}
\end{minipage}
\end{tabular}

\vspace{10mm}

% Course details

\textbf {\large \\Learning objectives:} Investigate a published fMRI study; 
collaborate on a software project; work with complex and large datasets; convolution
(hemodynamic modeling, smoothing); interpolation (slice time correction, image
resampling); optimization (registration, advanced statistics); basic linear
algebra (statistics).


\textbf {\large \\ Project overview:}
The semester project involves ``investigating'' a published result using the
analysis of functional Magnetic Resonance Imaging (MRI).\footnote{Exactly what
each team means by ``investigate'' will need to be defined by each team.  For
example, for one team this might mean a very careful reanalysis of the data
using the original methods as closely as possible.  However, this shouldn't
mean merely running existing scripts used in the original analysis; rather, the
team would need to reimplement the analysis scripts.  Another team might decide
to conduct a different analysis than used in the original study.  For instance,
the published work might use a parametric approach and the team project might
attempt to use a nonparametric technique such as permutation testing.  A third
team might focus on a careful validation of the modeling assumptions made by
the original analysis.}    Functional MRI (fMRI) allows scientists to localize
which parts of the brain are associated with specific cognitive tasks. There is
no expectation that you will have a background in neuroscience. You will learn
everything you need to know about the method in the course. The intention of
focusing on fMRI data is merely to provide a concrete problem domain that
exemplifies the types of programming and statistical challenges present in many
modern statistical applications.  Additionally, the weekly labs will prepare
you---through a series of guided exercises---with the basic skills and
background you need for the group project.\\

While you will learn the basic methods and tools during lecture and lab, you
will be expected to do additional research and reading in the course of working
on the group project.  For instance, you may need to use a specialized analysis
method or a Python package not covered in the lectures or labs.\\

As part of your final project grade, you will be required to work on your
project using GitHub's pull request and code review mechanism.  During lecture,
I will cover the exact workflow you will be expected to follow.  Please note
that as this course is explicitly about reproducibility and collaboration your
project grade will not be entirely based on your final report, but will also
reflect how well your group work is reproducible and how effectively you
collaborated using the techniques taught in the course (e.g., pull requests,
code review, testing, etc.)

\textbf {\large \\ Forming teams:}
Teams will be decided by \textbf{Thursday, September 17th}.  Each team
should consist of 4 students.  Students enrolled in 159 (259) will be required
to work with students enrolled in 159 (259).  You should ensure that your team
is composed of individuals with different strengths.  For example, each team
will need people with strong computational as well as statistical skills.

Prior to forming teams, I will ask each student to look over the list of
project ideas and indicate which of the published papers they would be
interested in investigating.  Then I will ask everyone interested in a
particular paper to meet together during class to discuss potential team
projects.

\textbf {\large \\ Project proposal:}
Each team will submit a project proposal on \textbf{Thursday, October
1st}.  The proposal should be written in \LaTeX and submitted to your team's
GitHub repository.  Use this template
\url{http://www.jarrodmillman.com/stat159-fall2015/project/proposal.tex}.

\textbf {\large \\ Progress presentation:}
Each team will present a 5 minute progress report in class
on \textbf{Tuesday, November 3rd} or \textbf{Thursday,  November 5th}.

\textbf {\large \\ Draft report:}
Each team will submit a written draft report by 21:00 \textbf{Thursday,
November 12th}.  For more information see the template
\url{http://www.jarrodmillman.com/stat159-fall2015/project/report.tex}.

\textbf {\large \\ Project presentation:}
Each team will present a 5 minute progress report in class
on \textbf{Tuesday, December 1st} or \textbf{Thursday, December 3rd}.

\textbf {\large \\ Final report:}
Each team will submit a final written report by 21:00 on
\textbf{Monday, December 14}.  For more information see the
template \url{http://www.jarrodmillman.com/stat159-fall2015/project/report.tex}.

\textbf {\large \\ Additional resources:}

\begin{itemize}
\item \url{https://www.coursera.org/course/fmri}
\end{itemize}

\end{document}
